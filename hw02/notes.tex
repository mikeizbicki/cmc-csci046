\documentclass[10pt]{article}
\usepackage[utf8]{inputenc}

\usepackage[margin=1in]{geometry}
\usepackage{amsmath}
\usepackage{amssymb}
\usepackage{amsthm}
\usepackage{mathtools}

\usepackage{hyperref}
\hypersetup{
  colorlinks   = true, %Colours links instead of ugly boxes
  urlcolor     = black, %Colour for external hyperlinks
  linkcolor    = blue, %Colour of internal links
  citecolor    = blue  %Colour of citations
}

\usepackage{color}
\definecolor{deepblue}{rgb}{0,0,0.5}
\definecolor{deepred}{rgb}{0.6,0,0}
\definecolor{deepgreen}{rgb}{0,0.5,0}

%%%%%%%%%%%%%%%%%%%%%%%%%%%%%%%%%%%%%%%%%%%%%%%%%%%%%%%%%%%%%%%%%%%%%%%%%%%%%%%%

\theoremstyle{definition}
\newtheorem{problem}{Problem}
\newtheorem{example}{Example}
\newcommand{\E}{\mathbb E}
\newcommand{\R}{\mathbb R}
\DeclareMathOperator{\Var}{Var}
\DeclareMathOperator*{\argmin}{arg\,min}
\DeclareMathOperator*{\argmax}{arg\,max}

\newcommand{\trans}[1]{{#1}^{T}}
\newcommand{\loss}{\ell}
\newcommand{\w}{\mathbf w}
\newcommand{\x}{\mathbf x}
\newcommand{\y}{\mathbf y}
\newcommand{\ltwo}[1]{\lVert {#1} \rVert}


\usepackage{listings}

% Default fixed font does not support bold face
\DeclareFixedFont{\ttb}{T1}{txtt}{bx}{n}{12} % for bold
\DeclareFixedFont{\ttm}{T1}{txtt}{m}{n}{12}  % for normal

% Python style for highlighting
\newcommand\pythonstyle{\lstset{
language=Python,
basicstyle=\ttm,
otherkeywords={self},             % Add keywords here
keywordstyle=\ttb\color{deepblue},
emph={MyClass,__init__},          % Custom highlighting
emphstyle=\ttb\color{deepred},    % Custom highlighting style
stringstyle=\color{deepgreen},
frame=tb,                         % Any extra options here
showstringspaces=false,           % 
stepnumber=1,
numbers=left
}}

\lstnewenvironment{python}[1][]
{
    \pythonstyle
    \lstset{#1}
}
{}

%%%%%%%%%%%%%%%%%%%%%%%%%%%%%%%%%%%%%%%%%%%%%%%%%%%%%%%%%%%%%%%%%%%%%%%%%%%%%%%%

\begin{document}

\begin{center}
    {
\Large
CSCI046 Notes: Runtime Analysis
}

    %\vspace{0.1in}
%CSCI046, Mike Izbicki
\end{center}

\vspace{0.25in}
\noindent

\begin{example}
Answer the questions below based on the following python code:
\begin{python}
print('x')
print('x')
print('x')

for i in range(10):
    print('y')

for i in range(10,20):
    print('z')
    print('z')
    print('z')
\end{python}
    \begin{enumerate}
        \item What is the exact number of times that the letter \texttt{x} will be printed?
            \vspace{1.5in}
        \item What is the exact number of times that the letter \texttt{y} will be printed?
            \vspace{1.5in}
        \item What is the exact number of times that the letter \texttt{z} will be printed?
        %\item How many times is the letter \texttt{x} printed?
            %\vspace{1.5in}
        %\item How many times is the letter \texttt{y} printed?
            %\vspace{1.5in}
    \end{enumerate}
\end{example}

\newpage
\begin{example}
Answer the questions below based on the following python code:
\begin{python}
for i in range(10):
    print('x')
for i in range(20):
    print('x')

for i in range(10):
    for j in range(20):
        for k in range(30):
            print('y')

print('z')
for i in range(10):
    print('z')
    for j in range(10):
        print('z')
        print('z')
    for j in range(10):
        print('z')
for i in range(10):
    print('z')
\end{python}
    \begin{enumerate}
        \item What is the exact number of times that the letter \texttt{x} will be printed?
            \vspace{1.5in}
        \item What is the exact number of times that the letter \texttt{y} will be printed?
            \vspace{1.5in}
        \item What is the exact number of times that the letter \texttt{z} will be printed?
        %\item How many times is the letter \texttt{x} printed?
            %\vspace{1.5in}
        %\item How many times is the letter \texttt{y} printed?
            %\vspace{1.5in}
        %\item How many times is the letter \texttt{z} printed?
            %\vspace{1.5in}
    \end{enumerate}
\end{example}

\newpage
\begin{example}
Answer the questions below based on the following python code:
\begin{python}
for i in range(n):
    print('x')
for i in range(n*2):
    print('x')

for i in range(n):
    for j in range(n*2):
        for k in range(n*3):
            print('y')

print('z')
for i in range(n):
    print('z')
    for j in range(n):
        print('z')
        print('z')
    for j in range(n):
        print('z')
for i in range(n):
    print('z')
\end{python}
    \begin{enumerate}
        \item What is the exact number of times that the letter \texttt{x} will be printed?
            \vspace{1.5in}
        \item What is the exact number of times that the letter \texttt{y} will be printed?
            \vspace{1.5in}
        \item What is the exact number of times that the letter \texttt{z} will be printed?
        %\item How many times is the letter \texttt{x} printed?
            %\vspace{1.5in}
        %\item How many times is the letter \texttt{y} printed?
            %\vspace{1.5in}
        %\item How many times is the letter \texttt{z} printed?
            %\vspace{1.5in}
    \end{enumerate}
\end{example}

\newpage
\begin{example}
Answer the questions below based on the following python code:
\begin{python}
for i in range(n):
    for j in range(0,i):
        for k in range(0,j):
            print('x')

for i in range(n):
    for j in range(i,n):
        for k in range(j,n):
            print('y')

for i in range(n):
    for j in range(i,n):
        for k in range(i,j):
            print('z')

\end{python}
    \begin{enumerate}
        \item What is the exact number of times that the letter \texttt{x} will be printed?
            \vspace{1.5in}
        \item What is the exact number of times that the letter \texttt{y} will be printed?
            \vspace{1.5in}
        \item What is the exact number of times that the letter \texttt{z} will be printed?
        %\item How many times is the letter \texttt{x} printed?
            %\vspace{1.5in}
        %\item How many times is the letter \texttt{y} printed?
            %\vspace{1.5in}
        %\item How many times is the letter \texttt{z} printed?
            %\vspace{1.5in}
    \end{enumerate}
\end{example}
\end{document}

