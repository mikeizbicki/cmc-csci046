\documentclass[10pt]{article}

\usepackage[margin=1in]{geometry}
\usepackage{amsmath}
\usepackage{amssymb}
\usepackage{amsthm}
\usepackage{mathtools}
\usepackage[shortlabels]{enumitem}
\usepackage[normalem]{ulem}
\usepackage{courier}

\usepackage{hyperref}
\hypersetup{
  colorlinks   = true, %Colours links instead of ugly boxes
  urlcolor     = black, %Colour for external hyperlinks
  linkcolor    = blue, %Colour of internal links
  citecolor    = blue  %Colour of citations
}

\usepackage[T1]{fontenc}
\usepackage{upquote}
\usepackage{listings}
\lstset{
    language=HTML
    ,basicstyle=\linespread{1}\ttfamily
    ,keywordstyle=
    ,language=sh
    ,showstringspaces=false
    ,numbers=left
    ,breaklines=true
    }


%%%%%%%%%%%%%%%%%%%%%%%%%%%%%%%%%%%%%%%%%%%%%%%%%%%%%%%%%%%%%%%%%%%%%%%%%%%%%%%%

\theoremstyle{definition}
\newtheorem{problem}{Problem}
\newtheorem{note}{Note}
\newcommand{\E}{\mathbb E}
\newcommand{\R}{\mathbb R}
\DeclareMathOperator{\Var}{Var}
\DeclareMathOperator*{\argmin}{arg\,min}
\DeclareMathOperator*{\argmax}{arg\,max}

\newcommand{\trans}[1]{{#1}^{T}}
\newcommand{\loss}{\ell}
\newcommand{\w}{\mathbf w}
\newcommand{\mle}[1]{\hat{#1}_{\textit{mle}}}
\newcommand{\map}[1]{\hat{#1}_{\textit{map}}}
\newcommand{\normal}{\mathcal{N}}
\newcommand{\x}{\mathbf x}
\newcommand{\y}{\mathbf y}
\newcommand{\ltwo}[1]{\lVert {#1} \rVert}

%%%%%%%%%%%%%%%%%%%%%%%%%%%%%%%%%%%%%%%%%%%%%%%%%%%%%%%%%%%%%%%%%%%%%%%%%%%%%%%%

\begin{document}
\begin{center}
    {
\Large
    Quiz: Recursion (Practice Problems)
}

    \vspace{0.1in}
\end{center}

\begin{note}
    You will have two quizzes on this material.
    The first quiz will be worth $2^2$ points,
    and the second quiz worth $2^3$ points.
    (Both will have $2^2$ problems,
    the problems in the second quiz will be worth 2 points each.)
\end{note}

\begin{note}
    All of these problems are designed to tests your ability to reason about recursion in an abstract setting.
    The problems in the \texttt{notes.py} file use recursion to actually implement the practical problem of binary search.
\end{note}

\begin{note}
    Recall that recursion must have a base case.
    Without a base case, the recursion will cause a \emph{stack overflow},
    which in python is represented by throwing a \texttt{RuntimeError}.
    The code below catches this exception.
\end{note}

\filbreak
\section{Basic Recursion}

\begin{problem}
    Write the output of the final command in the following terminal session.
    If the command has no output, then leave the problem blank.
\end{problem}
\begin{lstlisting}
$ cd; rm -rf quiz; mkdir quiz; cd quiz
$ cat > foo.py <<EOF
def foo(xs):
    if len(xs) == 0:
        return 0
    return xs[0] + foo(xs[1:])
try:
    print('foo([1, 2, 3, 4, 5])=',foo([1, 2, 3, 4, 5]))
except RuntimeError:
    print('StackOverflow')
EOF
$ python3 foo.py
\end{lstlisting}

\begin{problem}
    Write the output of the final command in the following terminal session.
    If the command has no output, then leave the problem blank.
\end{problem}
\begin{lstlisting}
$ cd; rm -rf quiz; mkdir quiz; cd quiz
$ cat > foo.py <<EOF
def foo(xs):
    ret = foo(xs[1:])
    if len(xs) == 0:
        return 0
    return xs[0] + ret
try:
    print('foo([1, 2, 3, 4, 5])=',foo([1, 2, 3, 4, 5]))
except RuntimeError:
    print('StackOverflow')
EOF
$ python3 foo.py
\end{lstlisting}

\filbreak
\begin{problem}
    Write the output of the final command in the following terminal session.
    If the command has no output, then leave the problem blank.
\end{problem}
\begin{lstlisting}
$ cd; rm -rf quiz; mkdir quiz; cd quiz
$ cat > foo.py <<EOF
def foo(xs):
    return foo(xs[1:]) + xs[0]
try:
    print('foo([1, 2, 3, 4, 5])=',foo([1, 2, 3, 4, 5]))
except RuntimeError:
    print('StackOverflow')
EOF
$ python3 foo.py
\end{lstlisting}


\filbreak
\begin{problem}
    Write the output of the final command in the following terminal session.
    If the command has no output, then leave the problem blank.
\end{problem}
\begin{lstlisting}
$ cd; rm -rf quiz; mkdir quiz; cd quiz
$ cat > foo.py <<EOF
def foo(xs):
    if len(xs) == 0:
        return 0
    return - foo(xs[1:]) + xs[0]
try:
    print('foo([1, 2, 3, 4, 5])=',foo([1, 2, 3, 4, 5]))
except RuntimeError:
    print('StackOverflow')
EOF
$ python3 foo.py
\end{lstlisting}


\filbreak
\begin{problem}
    Write the output of the final command in the following terminal session.
    If the command has no output, then leave the problem blank.
\end{problem}
\begin{lstlisting}
$ cd; rm -rf quiz; mkdir quiz; cd quiz
$ cat > foo.py <<EOF
def foo(xs):
    if len(xs) == 0:
        return 0
    return foo(xs[1:-1]) * xs[-1]
try:
    print('foo([1, 2, 3, 4, 5])=',foo([1, 2, 3, 4, 5]))
except RuntimeError:
    print('StackOverflow')
EOF
$ python3 foo.py
\end{lstlisting}

\filbreak
\begin{problem}
    Write the output of the final command in the following terminal session.
    If the command has no output, then leave the problem blank.
\end{problem}
\begin{lstlisting}
$ cd; rm -rf quiz; mkdir quiz; cd quiz
$ cat > foo.py <<EOF
def foo(xs):
    if len(xs) == 2:
        return 0
    return foo(xs[1:-1]) + xs[-1]
try:
    print('foo([1, 2, 3, 4, 5])=',foo([1, 2, 3, 4, 5]))
except RuntimeError:
    print('StackOverflow')
EOF
$ python3 foo.py
\end{lstlisting}


\filbreak
\begin{problem}
    Write the output of the final command in the following terminal session.
    If the command has no output, then leave the problem blank.
\end{problem}
\begin{lstlisting}
$ cd; rm -rf quiz; mkdir quiz; cd quiz
$ cat > foo.py <<EOF
def foo(xs):
    ret = foo(xs[1:-1]) + xs[-1]
    if len(xs) == 2:
        return 0
    return ret
try:
    print('foo([1, 2, 3, 4, 5])=',foo([1, 2, 3, 4, 5]))
except RuntimeError:
    print('StackOverflow')
EOF
$ python3 foo.py
\end{lstlisting}

\filbreak
\section{Using a Helper Function}

\begin{problem}
    Write the output of the final command in the following terminal session.
    If the command has no output, then leave the problem blank.
\end{problem}
\begin{lstlisting}
$ cd; rm -rf quiz; mkdir quiz; cd quiz
$ cat > foo.py <<EOF
def foo(xs):
    def go(i):
        if i == len(xs):
            return 0
        return xs[i] + go(i+1)
    return go(0)
try:
    print('foo([1, 2, 3, 4, 5])=',foo([1, 2, 3, 4, 5]))
except RuntimeError:
    print('StackOverflow')
EOF
$ python3 foo.py
\end{lstlisting}

\filbreak
\begin{problem}
    Write the output of the final command in the following terminal session.
    If the command has no output, then leave the problem blank.
\end{problem}
\begin{lstlisting}
$ cd; rm -rf quiz; mkdir quiz; cd quiz
$ cat > foo.py <<EOF
def foo(xs):
    def go(i):
        if i == -1:
            return 0
        return xs[i] + go(i-1)
    return go(len(xs)-1)
try:
    print('foo([1, 2, 3, 4, 5])=',foo([1, 2, 3, 4, 5]))
except RuntimeError:
    print('StackOverflow')
EOF
$ python3 foo.py
\end{lstlisting}


\filbreak
\begin{problem}
    Write the output of the final command in the following terminal session.
    If the command has no output, then leave the problem blank.
\end{problem}
\begin{lstlisting}
$ cd; rm -rf quiz; mkdir quiz; cd quiz
$ cat > foo.py <<EOF
def foo(xs):
    def go(i):
        if i == 0:
            return xs[0]
        return xs[i] + go(i // 2)
    return go(len(xs)-1)
try:
    print('foo([1, 2, 3, 4, 5])=',foo([1, 2, 3, 4, 5]))
except RuntimeError:
    print('StackOverflow')
EOF
$ python3 foo.py
\end{lstlisting}


\filbreak
\begin{problem}
    Write the output of the final command in the following terminal session.
    If the command has no output, then leave the problem blank.
\end{problem}
\begin{lstlisting}
$ cd; rm -rf quiz; mkdir quiz; cd quiz
$ cat > foo.py <<EOF
def foo(xs):
    if len(xs) == 0:
        return 0
    def go(i):
        return xs[i] + go(i+1)
    return go(0)
try:
    print('foo([1, 2, 3, 4, 5])=',foo([1, 2, 3, 4, 5]))
except RuntimeError:
    print('StackOverflow')
EOF
$ python3 foo.py
\end{lstlisting}

\filbreak
\section{With an accumulator}

\begin{problem}
    Write the output of the final command in the following terminal session.
    If the command has no output, then leave the problem blank.
\end{problem}
\begin{lstlisting}
$ cd; rm -rf quiz; mkdir quiz; cd quiz
$ cat > foo.py <<EOF
def foo(xs):
    def go(i, acc):
        if len(xs) == i:
            return acc
        return go(i+1, acc + xs[i])
    return go(0, 0)
try:
    print('foo([1, 2, 3, 4, 5])=',foo([1, 2, 3, 4, 5]))
except RuntimeError:
    print('StackOverflow')
EOF
$ python3 foo.py
\end{lstlisting}

\filbreak
\begin{problem}
    Write the output of the final command in the following terminal session.
    If the command has no output, then leave the problem blank.
\end{problem}
\begin{lstlisting}
$ cd; rm -rf quiz; mkdir quiz; cd quiz
$ cat > foo.py <<EOF
def foo(xs):
    def go(i, acc):
        if len(xs) == i:
            return acc
        return go(i+1, acc + xs[i])
    return go(1, 2)
try:
    print('foo([1, 2, 3, 4, 5])=',foo([1, 2, 3, 4, 5]))
except RuntimeError:
    print('StackOverflow')
EOF
$ python3 foo.py
\end{lstlisting}


\filbreak
\begin{problem}
    Write the output of the final command in the following terminal session.
    If the command has no output, then leave the problem blank.
\end{problem}
\begin{lstlisting}
$ cd; rm -rf quiz; mkdir quiz; cd quiz
$ cat > foo.py <<EOF
def foo(xs):
    def go(i, acc):
        if len(xs) == i:
            return acc
        return go(i+1, - acc + xs[i])
    return go(0, 0)
try:
    print('foo([1, 2, 3, 4, 5])=',foo([1, 2, 3, 4, 5]))
except RuntimeError:
    print('StackOverflow')
EOF
$ python3 foo.py
\end{lstlisting}

\section{Multiple Recursion}

\begin{problem}
    Write the output of the final command in the following terminal session.
    If the command has no output, then leave the problem blank.
\end{problem}
\begin{lstlisting}
$ cd; rm -rf quiz; mkdir quiz; cd quiz
$ cat > foo.py <<EOF
def foo(xs):
    if len(xs) == 0:
        return 0
    return xs[0] + foo(xs[1:]) + foo(xs[2:])
try:
    print('foo([1, 2, 3])=',foo([1, 2, 3]))
except RuntimeError:
    print('StackOverflow')
EOF
$ python3 foo.py
\end{lstlisting}


\begin{problem}
    Write the output of the final command in the following terminal session.
    If the command has no output, then leave the problem blank.
\end{problem}
\begin{lstlisting}
$ cd; rm -rf quiz; mkdir quiz; cd quiz
$ cat > foo.py <<EOF
def foo(xs):
    if len(xs) == 0:
        return 0
    ret = xs[0]
    ret += foo(xs[1:])
    ret += foo(xs[:-1])
    return ret
try:
    print('foo([1, 2, 3])=',foo([1, 2, 3]))
except RuntimeError:
    print('StackOverflow')
EOF
$ python3 foo.py
\end{lstlisting}

\filbreak
\begin{problem}
    Write the output of the final command in the following terminal session.
    If the command has no output, then leave the problem blank.
\end{problem}
\begin{lstlisting}
$ cd; rm -rf quiz; mkdir quiz; cd quiz
$ cat > foo.py <<EOF
def foo(xs):
    def go(i, acc):
        if len(xs) <= i:
            return acc
        ret = 0
        ret += go(i+1, acc + xs[i])
        ret += go(i+2, acc + xs[i])
        ret += go(i+3, acc + xs[i])
        return ret
    return go(0, 0)
try:
    print('foo([1, 2, 3])=',foo([1, 2, 3]))
except RuntimeError:
    print('StackOverflow')
EOF
$ python3 foo.py
\end{lstlisting}


\filbreak
\begin{problem}
    Write the output of the final command in the following terminal session.
    If the command has no output, then leave the problem blank.
\end{problem}
\begin{lstlisting}
$ cd; rm -rf quiz; mkdir quiz; cd quiz
$ cat > foo.py <<EOF
def foo(xs):
    def go(i, acc):
        if len(xs) <= i:
            return acc
        ret = 0
        ret += go(i+1, acc + xs[i])
        ret += go(i+1, acc + xs[i])
        return ret
    return go(0, 0)
try:
    print('foo([1, 2, 3])=',foo([1, 2, 3]))
except RuntimeError:
    print('StackOverflow')
EOF
$ python3 foo.py
\end{lstlisting}

\filbreak
\section{Building Lists with Recursion}

\begin{problem}
    Write the output of the final command in the following terminal session.
    If the command has no output, then leave the problem blank.
\end{problem}
\begin{lstlisting}
$ cd; rm -rf quiz; mkdir quiz; cd quiz
$ cat > foo.py <<EOF
def foo(xs):
    if len(xs) == 10:
        return 1
    xs.append(1)
    return 2 * foo(xs)
try:
    print('foo([])=',foo([]))
except RuntimeError:
    print('StackOverflow')
EOF
$ python3 foo.py
\end{lstlisting}


\filbreak
\begin{problem}
    Write the output of the final command in the following terminal session.
    If the command has no output, then leave the problem blank.
\end{problem}
\begin{lstlisting}
$ cd; rm -rf quiz; mkdir quiz; cd quiz
$ cat > foo.py <<EOF
def foo(xs):
    if len(xs) == 10:
        return 1
    xs.append(1)
    return 2 * foo(xs)
try:
    print('foo([1, 2, 3])=',foo([1, 2, 3]))
except RuntimeError:
    print('StackOverflow')
EOF
$ python3 foo.py
\end{lstlisting}

\filbreak
\begin{problem}
    Write the output of the final command in the following terminal session.
    If the command has no output, then leave the problem blank.
\end{problem}
\begin{lstlisting}
$ cd; rm -rf quiz; mkdir quiz; cd quiz
$ cat > foo.py <<EOF
def foo(xs):
    xs.append(1)
    return 2 * foo(xs)
    if len(xs) == 10:
        return 1
try:
    print('foo([1, 2, 3])=',foo([1, 2, 3]))
except RuntimeError:
    print('StackOverflow')
EOF
$ python3 foo.py
\end{lstlisting}


\end{document}
