\documentclass[10pt]{article}

\usepackage[margin=1in]{geometry}
\usepackage{amsmath}
\usepackage{amssymb}
\usepackage{amsthm}
\usepackage{mathtools}
\usepackage[shortlabels]{enumitem}
\usepackage[normalem]{ulem}
\usepackage{courier}

\usepackage{hyperref}
\hypersetup{
  colorlinks   = true, %Colours links instead of ugly boxes
  urlcolor     = black, %Colour for external hyperlinks
  linkcolor    = blue, %Colour of internal links
  citecolor    = blue  %Colour of citations
}

\usepackage[T1]{fontenc}
\usepackage{upquote}
\usepackage{listings}
\lstset{
    language=HTML
    ,basicstyle=\linespread{1}\ttfamily
    ,keywordstyle=
    ,language=sh
    ,showstringspaces=false
    ,numbers=left
    ,breaklines=true
    }

%%%%%%%%%%%%%%%%%%%%%%%%%%%%%%%%%%%%%%%%%%%%%%%%%%%%%%%%%%%%%%%%%%%%%%%%%%%%%%%%

\theoremstyle{definition}
\newtheorem{problem}{Problem}
\newtheorem{note}{Note}
\newcommand{\E}{\mathbb E}
\newcommand{\R}{\mathbb R}
\DeclareMathOperator{\Var}{Var}
\DeclareMathOperator*{\argmin}{arg\,min}
\DeclareMathOperator*{\argmax}{arg\,max}

\newcommand{\trans}[1]{{#1}^{T}}
\newcommand{\loss}{\ell}
\newcommand{\w}{\mathbf w}
\newcommand{\mle}[1]{\hat{#1}_{\textit{mle}}}
\newcommand{\map}[1]{\hat{#1}_{\textit{map}}}
\newcommand{\normal}{\mathcal{N}}
\newcommand{\x}{\mathbf x}
\newcommand{\y}{\mathbf y}
\newcommand{\ltwo}[1]{\lVert {#1} \rVert}

%%%%%%%%%%%%%%%%%%%%%%%%%%%%%%%%%%%%%%%%%%%%%%%%%%%%%%%%%%%%%%%%%%%%%%%%%%%%%%%%

\begin{document}
\begin{center}
    {
\Large
    Quiz: OOP (2)
}

    %\vspace{0.1in}
    %CSCI046: Data Structures

    \vspace{0.1in}
\end{center}

\vspace{0.15in}
\noindent
\textbf{Total Score:} ~~~~~~~~~~~~~~~/$2^2$

\vspace{0.2in}
\noindent
\textbf{Printed Name:}

\noindent
\rule{\textwidth}{0.1pt}
\vspace{0.15in}

\noindent
\textbf{Quiz rules:}
\begin{enumerate}
    \item You MAY use any printed or handwritten notes.
    \item You MAY NOT use a computer or any other electronic device.
\end{enumerate}

\noindent

\vspace{0.15in}

\begin{problem}
    Write the output of the final command in the following terminal session.
    If the command has no output, then leave the problem blank.
\end{problem}
\begin{lstlisting}
$ cd; rm -rf quiz; mkdir quiz; cd quiz
$ cat > foo.py <<EOF
class Foo:
    def __init__(self, message=None):
        if message:
            Foo.message = message
foo = Foo('hello world')
foo.child = Foo()
foo.child.child = foo
try:
    print(' foo.child.child.child.message=', foo.child.child.child.message)
except AttributeError:
    print('AttributeError') 
EOF
$ python3 foo.py
\end{lstlisting}

\newpage
\begin{problem}
    Write the output of the final command in the following terminal session.
    If the command has no output, then leave the problem blank.
\end{problem}
\begin{lstlisting}
$ cd; rm -rf quiz; mkdir quiz; cd quiz
$ cat > foo.py <<EOF
class Foo:
    def __init__(self, message=''):
        self.message = message
    def foo(self):
        return self.message
class Bar(Foo):
    def __init__(self, message=''):
        super().__init__(message)
    def bar(self):
        return self.message
a = Foo('hello world')
b = Bar('hola mundo')
try:
    print('a.bar()=', a.bar())
except AttributeError:
    print('AttributeError') 
EOF
$ python3 foo.py
\end{lstlisting}


\newpage
\begin{problem}
    Write the output of the final command in the following terminal session.
    If the command has no output, then leave the problem blank.
\end{problem}
\begin{lstlisting}
$ cd; rm -rf quiz; mkdir quiz; cd quiz
$ cat > foo.py <<EOF
class Foo:
    message = 'salve munde'
    def __init__(self, message=None):
        if message:
            Foo.message = message
    @staticmethod
    def foo():
        return Foo.message
class Bar(Foo):
    def __init__(self, message=None):
        super().__init__(message)
    @staticmethod
    def bar():
        return Bar.message
a = Foo('hello world')
b = Bar()
try:
    print('b.bar()=', b.bar())
except AttributeError:
    print('AttributeError') 
EOF
$ python3 foo.py
\end{lstlisting}

\newpage
\begin{problem}
    Write the output of the final command in the following terminal session.
    If the command has no output, then leave the problem blank.
\end{problem}
\begin{lstlisting}
$ cd; rm -rf quiz; mkdir quiz; cd quiz
$ cat > foo.py <<EOF
class Foo(list):
    def __init__(self, xs=[]):
        super().__init__(xs)
        xs.append(len(xs))
xs = Foo()
xs.append(Foo())
xs.append(Foo())
ys = Foo()
ys.append(Foo())
ys.append(Foo())
try:
    print('len(ys[2])=', len(ys[2]))
except AttributeError:
    print('AttributeError') 
EOF
$ python3 foo.py
\end{lstlisting}

\end{document}
