\documentclass[10pt]{article}

\usepackage[margin=1in]{geometry}
\usepackage{amsmath}
\usepackage{amssymb}
\usepackage{amsthm}
\usepackage{mathtools}
\usepackage[shortlabels]{enumitem}
\usepackage[normalem]{ulem}
\usepackage{courier}

\usepackage{hyperref}
\hypersetup{
  colorlinks   = true, %Colours links instead of ugly boxes
  urlcolor     = black, %Colour for external hyperlinks
  linkcolor    = blue, %Colour of internal links
  citecolor    = blue  %Colour of citations
}

\usepackage[T1]{fontenc}
\usepackage{upquote}
\usepackage{listings}
\lstset{
    language=HTML
    ,basicstyle=\linespread{1}\ttfamily
    ,keywordstyle=
    ,language=sh
    ,showstringspaces=false
    ,numbers=left
    ,breaklines=true
    }


%%%%%%%%%%%%%%%%%%%%%%%%%%%%%%%%%%%%%%%%%%%%%%%%%%%%%%%%%%%%%%%%%%%%%%%%%%%%%%%%

\theoremstyle{definition}
\newtheorem{problem}{Problem}
\newtheorem{note}{Note}
\newcommand{\E}{\mathbb E}
\newcommand{\R}{\mathbb R}
\DeclareMathOperator{\Var}{Var}
\DeclareMathOperator*{\argmin}{arg\,min}
\DeclareMathOperator*{\argmax}{arg\,max}

\newcommand{\trans}[1]{{#1}^{T}}
\newcommand{\loss}{\ell}
\newcommand{\w}{\mathbf w}
\newcommand{\mle}[1]{\hat{#1}_{\textit{mle}}}
\newcommand{\map}[1]{\hat{#1}_{\textit{map}}}
\newcommand{\normal}{\mathcal{N}}
\newcommand{\x}{\mathbf x}
\newcommand{\y}{\mathbf y}
\newcommand{\ltwo}[1]{\lVert {#1} \rVert}

%%%%%%%%%%%%%%%%%%%%%%%%%%%%%%%%%%%%%%%%%%%%%%%%%%%%%%%%%%%%%%%%%%%%%%%%%%%%%%%%

\begin{document}
\begin{center}
    {
\Large
    Quiz: OOP II (Practice Problems)
}

    \vspace{0.1in}
\end{center}

\filbreak
\section{Inheritance}

\begin{note}
    Whenever an attribute is not found in a subclass,
    Python will check the superclass for the attribute before raising the \lstinline{AttributeError} exception.
\end{note}

\filbreak
\begin{problem}
    Write the output of the final command in the following terminal session.
    If the command has no output, then leave the problem blank.
\end{problem}
\begin{lstlisting}
$ cd; rm -rf quiz; mkdir quiz; cd quiz
$ cat > foo.py <<EOF
class Foo:
    pass
class Bar(Foo):
    pass
a = Foo()
Foo.message = 'hello world'
b = Bar()
Bar.message = 'hola mundo'
try:
    print('b.message=', b.message)
except AttributeError:
    print('AttributeError') 
EOF
$ python3 foo.py
\end{lstlisting}


\filbreak
\begin{problem}
    Write the output of the final command in the following terminal session.
    If the command has no output, then leave the problem blank.
\end{problem}
\begin{lstlisting}
$ cd; rm -rf quiz; mkdir quiz; cd quiz
$ cat > foo.py <<EOF
class Foo:
    pass
class Bar(Foo):
    pass
a = Foo()
Foo.message = 'hello world'
b = Bar()
a.message = 'hola mundo'
try:
    print('b.message=', b.message)
except AttributeError:
    print('AttributeError') 
EOF
$ python3 foo.py
\end{lstlisting}


\filbreak
\begin{problem}
    Write the output of the final command in the following terminal session.
    If the command has no output, then leave the problem blank.
\end{problem}
\begin{lstlisting}
$ cd; rm -rf quiz; mkdir quiz; cd quiz
$ cat > foo.py <<EOF
class Foo:
    pass
class Bar(Foo):
    pass
a = Foo()
Foo.message = 'hello world'
b = Bar()
a.message = 'hola mundo'
try:
    print('b.message=', b.message)
except AttributeError:
    print('AttributeError') 
EOF
$ python3 foo.py
\end{lstlisting}


\filbreak
\begin{problem}
    Write the output of the final command in the following terminal session.
    If the command has no output, then leave the problem blank.
\end{problem}
\begin{lstlisting}
$ cd; rm -rf quiz; mkdir quiz; cd quiz
$ cat > foo.py <<EOF
class Foo:
    pass
class Bar(Foo):
    pass
a = Foo()
a.message = 'hello world'
b = Bar()
try:
    print('b.message=', b.message)
except AttributeError:
    print('AttributeError') 
EOF
$ python3 foo.py
\end{lstlisting}

\filbreak
\begin{problem}
    Write the output of the final command in the following terminal session.
    If the command has no output, then leave the problem blank.
\end{problem}
\begin{lstlisting}
$ cd; rm -rf quiz; mkdir quiz; cd quiz
$ cat > foo.py <<EOF
class Foo:
    pass
class Bar(Foo):
    pass
a = Foo()
Foo.message = 'hello world'
b = Bar()
Bar.message = 'hola mundo'
try:
    print('a.message=', a.message)
except AttributeError:
    print('AttributeError') 
EOF
$ python3 foo.py
\end{lstlisting}

\subsection{With Constructors}

\begin{note}
    The constructor of a superclass will only be called if it is explicitly called in the constructor of the subclass.
    You should use the \lstinline{super} function to get the superclass.
\end{note}

\filbreak
\begin{problem}
    Write the output of the final command in the following terminal session.
    If the command has no output, then leave the problem blank.
\end{problem}
\begin{lstlisting}
$ cd; rm -rf quiz; mkdir quiz; cd quiz
$ cat > foo.py <<EOF
class Foo:
    def __init__(self):
        self.message = 'hello world'
class Bar(Foo):
    def __init__(self):
        self.message = 'hola mundo'
a = Foo()
b = Bar()
try:
    print('b.message=', b.message)
except AttributeError:
    print('AttributeError') 
EOF
$ python3 foo.py
\end{lstlisting}

\filbreak
\begin{problem}
    Write the output of the final command in the following terminal session.
    If the command has no output, then leave the problem blank.
\end{problem}
\begin{lstlisting}
$ cd; rm -rf quiz; mkdir quiz; cd quiz
$ cat > foo.py <<EOF
class Foo:
    def __init__(self):
        self.message = 'hello world'
class Bar(Foo):
    def __init__(self):
        self.message = 'hola mundo'
        super().__init__()
a = Foo()
b = Bar()
try:
    print('b.message=', b.message)
except AttributeError:
    print('AttributeError') 
EOF
$ python3 foo.py
\end{lstlisting}


\filbreak
\begin{problem}
    Write the output of the final command in the following terminal session.
    If the command has no output, then leave the problem blank.
\end{problem}
\begin{lstlisting}
$ cd; rm -rf quiz; mkdir quiz; cd quiz
$ cat > foo.py <<EOF
class Foo:
    def __init__(self):
        self.message = 'hello world'
class Bar(Foo):
    def __init__(self):
        super().__init__()
        self.message = 'hola mundo'
a = Foo()
b = Bar()
try:
    print('b.message=', b.message)
except AttributeError:
    print('AttributeError') 
EOF
$ python3 foo.py
\end{lstlisting}

\filbreak
\begin{problem}
    Write the output of the final command in the following terminal session.
    If the command has no output, then leave the problem blank.
\end{problem}
\begin{lstlisting}
$ cd; rm -rf quiz; mkdir quiz; cd quiz
$ cat > foo.py <<EOF
class Foo:
    def __init__(self, message=None):
        self.message = message
class Bar(Foo):
    def __init__(self, message=None):
        self.message = message
        super().__init__(message)
a = Foo('hello world')
b = Bar('hola mundo')
try:
    print('b.message=', b.message)
except AttributeError:
    print('AttributeError') 
EOF
$ python3 foo.py
\end{lstlisting}


\filbreak
\begin{problem}
    Write the output of the final command in the following terminal session.
    If the command has no output, then leave the problem blank.
\end{problem}
\begin{lstlisting}
$ cd; rm -rf quiz; mkdir quiz; cd quiz
$ cat > foo.py <<EOF
class Foo:
    def __init__(self, message=None):
        Foo.message = message
class Bar(Foo):
    def __init__(self, message=None):
        super().__init__(message)
a = Foo('hello world')
b = Bar('hola mundo')
try:
    print('b.message=', b.message)
except AttributeError:
    print('AttributeError') 
EOF
$ python3 foo.py
\end{lstlisting}


\filbreak
\begin{problem}
    Write the output of the final command in the following terminal session.
    If the command has no output, then leave the problem blank.
\end{problem}
\begin{lstlisting}
$ cd; rm -rf quiz; mkdir quiz; cd quiz
$ cat > foo.py <<EOF
class Foo:
    def __init__(self, message=None):
        Foo.message = message
class Bar(Foo):
    def __init__(self, message=None):
        super().__init__(message)
b = Bar('hola mundo')
a = Foo('hello world')
try:
    print('b.message=', b.message)
except AttributeError:
    print('AttributeError') 
EOF
$ python3 foo.py
\end{lstlisting}


\filbreak
\begin{problem}
    Write the output of the final command in the following terminal session.
    If the command has no output, then leave the problem blank.
\end{problem}
\begin{lstlisting}
$ cd; rm -rf quiz; mkdir quiz; cd quiz
$ cat > foo.py <<EOF
class Foo:
    pass
class Bar(Foo):
    def __init__(self, message=None):
        super().__init__()
        Foo.message = message
a = Foo()
b = Bar('hola mundo')
c = Bar()
try:
    print('b.message=', b.message)
except AttributeError:
    print('AttributeError') 
EOF
$ python3 foo.py
\end{lstlisting}

\subsection{With built-in Classes}

\begin{note}
    Recall that when you subclasss an existing class, you get all the superclass's functionality "for free".
    It is very common in python to subclass the built-in python classes in order to extend their functionality.
    Again, these problems focus on the "what/how" instead of the "why".
\end{note}

\filbreak
\begin{problem}
    Write the output of the final command in the following terminal session.
    If the command has no output, then leave the problem blank.
\end{problem}
\begin{lstlisting}
$ cd; rm -rf quiz; mkdir quiz; cd quiz
$ cat > foo.py <<EOF
class Foo(list):
    def __init__(self, xs=[]):
        super().__init__(xs + [0])
xs = Foo([1])
xs = Foo([1, 2])
xs = Foo([1, 2, 3])
try:
    print('xs=', xs)
except AttributeError:
    print('AttributeError') 
EOF
$ python3 foo.py
\end{lstlisting}

\filbreak
\begin{problem}
    Write the output of the final command in the following terminal session.
    If the command has no output, then leave the problem blank.
\end{problem}
\begin{lstlisting}
$ cd; rm -rf quiz; mkdir quiz; cd quiz
$ cat > foo.py <<EOF
class Foo(list):
    def __init__(self, xs=[]):
        xs.append(len(xs))
        super().__init__(xs)
xs = Foo([1, 2, 3])
xs = Foo(xs)
xs = Foo(xs)
try:
    print('xs=', xs)
except AttributeError:
    print('AttributeError') 
EOF
$ python3 foo.py
\end{lstlisting}

\filbreak
\begin{problem}
    Write the output of the final command in the following terminal session.
    If the command has no output, then leave the problem blank.
\end{problem}
\begin{lstlisting}
$ cd; rm -rf quiz; mkdir quiz; cd quiz
$ cat > foo.py <<EOF
class Foo(list):
    def __init__(self, xs=[]):
        super().__init__(xs)
        xs.append(len(xs))
xs = Foo()
xs = Foo()
xs = Foo()
try:
    print('xs=', xs)
except AttributeError:
    print('AttributeError') 
EOF
$ python3 foo.py
\end{lstlisting}

\filbreak
\begin{problem}
    Write the output of the final command in the following terminal session.
    If the command has no output, then leave the problem blank.
\end{problem}
\begin{lstlisting}
$ cd; rm -rf quiz; mkdir quiz; cd quiz
$ cat > foo.py <<EOF
class Foo(list):
    def __init__(self, xs=[]):
        xs.append(len(xs))
        super().__init__(xs)
xs = Foo()
xs = Foo([1, 2, 3])
xs = Foo()
try:
    print('xs=', xs)
except AttributeError:
    print('AttributeError') 
EOF
$ python3 foo.py
\end{lstlisting}

%%%%%%%%%%%%%%%%%%%%%%%%%%%%%%%%%%%%%%%%%%%%%%%%%%%%%%%%%%%%%%%%%%%%%%%%%%%%%%%%

\section{Static Methods}

\subsection{Without Inheritance}

\filbreak
\begin{problem}
    Write the output of the final command in the following terminal session.
    If the command has no output, then leave the problem blank.
\end{problem}
\begin{lstlisting}
$ cd; rm -rf quiz; mkdir quiz; cd quiz
$ cat > foo.py <<EOF
class Foo:
    def __init__(self, message=None):
        self.message = message
    def foo(self):
        return self.message
a = Foo('hello world')
try:
    print('a.foo()=', a.foo())
except AttributeError:
    print('AttributeError') 
EOF
$ python3 foo.py
\end{lstlisting}

\filbreak
\begin{problem}
    Write the output of the final command in the following terminal session.
    If the command has no output, then leave the problem blank.
\end{problem}
\begin{lstlisting}
$ cd; rm -rf quiz; mkdir quiz; cd quiz
$ cat > foo.py <<EOF
class Foo:
    message = 'salve munde'
    def __init__(self, message=None):
        self.message = message
    @staticmethod
    def foo():
        return Foo.message
a = Foo('hello world')
try:
    print('a.foo()=', a.foo())
except AttributeError:
    print('AttributeError') 
EOF
$ python3 foo.py
\end{lstlisting}


\filbreak
\begin{problem}
    Write the output of the final command in the following terminal session.
    If the command has no output, then leave the problem blank.
\end{problem}
\begin{lstlisting}
$ cd; rm -rf quiz; mkdir quiz; cd quiz
$ cat > foo.py <<EOF
class Foo:
    message = 'salve munde'
    def __init__(self, message=None):
        self.message = message
    @staticmethod
    def foo():
        return Foo.message
a = Foo('hello world')
try:
    print('Foo.foo()=', Foo.foo())
except AttributeError:
    print('AttributeError') 
EOF
$ python3 foo.py
\end{lstlisting}


\filbreak
\begin{problem}
    Write the output of the final command in the following terminal session.
    If the command has no output, then leave the problem blank.
\end{problem}
\begin{lstlisting}
$ cd; rm -rf quiz; mkdir quiz; cd quiz
$ cat > foo.py <<EOF
class Foo:
    message = 'salve munde'
    def __init__(self, message=None):
        Foo.message = message
    @staticmethod
    def foo():
        return Foo.message
a = Foo('hello world')
try:
    print('Foo.foo()=', Foo.foo())
except AttributeError:
    print('AttributeError') 
EOF
$ python3 foo.py
\end{lstlisting}

\subsection{With Inheritance}

\filbreak
\begin{problem}
    Write the output of the final command in the following terminal session.
    If the command has no output, then leave the problem blank.
\end{problem}
\begin{lstlisting}
$ cd; rm -rf quiz; mkdir quiz; cd quiz
$ cat > foo.py <<EOF
class Foo:
    def __init__(self, message=None):
        self.message = message
    def foo(self):
        return self.message
class Bar(Foo):
    def __init__(self, message=None):
        super().__init__(message)
    def bar(self):
        return self.message
a = Foo('hello world')
b = Bar('hola mundo')
try:
    print('b.foo()=', b.foo())
except AttributeError:
    print('AttributeError') 
EOF
$ python3 foo.py
\end{lstlisting}


\filbreak
\begin{problem}
    Write the output of the final command in the following terminal session.
    If the command has no output, then leave the problem blank.
\end{problem}
\begin{lstlisting}
$ cd; rm -rf quiz; mkdir quiz; cd quiz
$ cat > foo.py <<EOF
class Foo:
    def __init__(self, message=None):
        self.message = message
    def foo(self):
        return self.message
class Bar(Foo):
    def __init__(self, message=None):
        super().__init__(message)
    def bar(self):
        return self.message
a = Foo('hello world')
b = Bar('hola mundo')
try:
    print('a.bar()=', a.bar())
except AttributeError:
    print('AttributeError') 
EOF
$ python3 foo.py
\end{lstlisting}


\filbreak
\begin{problem}
    Write the output of the final command in the following terminal session.
    If the command has no output, then leave the problem blank.
\end{problem}
\begin{lstlisting}
$ cd; rm -rf quiz; mkdir quiz; cd quiz
$ cat > foo.py <<EOF
class Foo:
    def __init__(self, message=None):
        self.message = message
    def foo(self):
        return self.message
class Bar(Foo):
    message = 'salve munde'
    def __init__(self, message=None):
        super().__init__(message)
    @staticmethod
    def bar():
        return Bar.message
a = Foo('hello world')
b = Bar('hola mundo')
try:
    print('b.bar()=', b.bar())
except AttributeError:
    print('AttributeError') 
EOF
$ python3 foo.py
\end{lstlisting}


\filbreak
\begin{problem}
    Write the output of the final command in the following terminal session.
    If the command has no output, then leave the problem blank.
\end{problem}
\begin{lstlisting}
$ cd; rm -rf quiz; mkdir quiz; cd quiz
$ cat > foo.py <<EOF
class Foo:
    def __init__(self, message=None):
        self.message = message
    def foo(self):
        return self.message
class Bar(Foo):
    message = 'salve munde'
    def __init__(self, message=None):
        super().__init__(message)
    @staticmethod
    def bar():
        return Bar.message
a = Foo('hello world')
b = Bar('hola mundo')
try:
    print('Bar.bar()=', Bar.bar())
except AttributeError:
    print('AttributeError') 
EOF
$ python3 foo.py
\end{lstlisting}


\filbreak
\begin{problem}
    Write the output of the final command in the following terminal session.
    If the command has no output, then leave the problem blank.
\end{problem}
\begin{lstlisting}
$ cd; rm -rf quiz; mkdir quiz; cd quiz
$ cat > foo.py <<EOF
class Foo:
    message = 'salve munde'
    def __init__(self, message=None):
        Bar.message = message
    @staticmethod
    def foo():
        return Bar.message
class Bar(Foo):
    def __init__(self, message=None):
        super().__init__(message)
    @staticmethod
    def bar():
        return Foo.message
a = Foo('hello world')
b = Bar('hola mundo')
try:
    print('Bar.foo()=', Bar.foo())
except AttributeError:
    print('AttributeError') 
EOF
$ python3 foo.py
\end{lstlisting}


\filbreak
\begin{problem}
    Write the output of the final command in the following terminal session.
    If the command has no output, then leave the problem blank.
\end{problem}
\begin{lstlisting}
$ cd; rm -rf quiz; mkdir quiz; cd quiz
$ cat > foo.py <<EOF
class Foo:
    message = 'salve munde'
    def __init__(self, message=None):
        Bar.message = message
    @staticmethod
    def foo():
        return Bar.message
class Bar(Foo):
    def __init__(self, message=None):
        super().__init__(message)
    @staticmethod
    def bar():
        return Foo.message
a = Foo('hello world')
b = Bar('hola mundo')
try:
    print('Bar.bar()=', Bar.bar())
except AttributeError:
    print('AttributeError') 
EOF
$ python3 foo.py
\end{lstlisting}

%%%%%%%%%%%%%%%%%%%%%%%%%%%%%%%%%%%%%%%%%%%%%%%%%%%%%%%%%%%%%%%%%%%%%%%%%%%%%%%%

\section{Recursive Classes}

\filbreak
\begin{problem}
    Write the output of the final command in the following terminal session.
    If the command has no output, then leave the problem blank.
\end{problem}
\begin{lstlisting}
$ cd; rm -rf quiz; mkdir quiz; cd quiz
$ cat > foo.py <<EOF
class Foo:
    def __init__(self, message=None):
        self.message = message
foo = Foo('hello world')
foo.child = Foo('hola mundo')
foo.child.child = Foo('salve munde')
try:
    print(' foo.child.message=', foo.child.message)
except AttributeError:
    print('AttributeError') 
EOF
$ python3 foo.py
\end{lstlisting}

\filbreak
\begin{problem}
    Write the output of the final command in the following terminal session.
    If the command has no output, then leave the problem blank.
\end{problem}
\begin{lstlisting}
$ cd; rm -rf quiz; mkdir quiz; cd quiz
$ cat > foo.py <<EOF
class Foo:
    def __init__(self, message=None):
        self.message = message
foo = Foo('hello world')
foo.child = Foo('hola mundo')
foo.child.child = Foo('salve munde')
try:
    print(' foo.child.child.child.message=', foo.child.child.child.message)
except AttributeError:
    print('AttributeError') 
EOF
$ python3 foo.py
\end{lstlisting}


\filbreak
\begin{problem}
    Write the output of the final command in the following terminal session.
    If the command has no output, then leave the problem blank.
\end{problem}
\begin{lstlisting}
$ cd; rm -rf quiz; mkdir quiz; cd quiz
$ cat > foo.py <<EOF
class Foo:
    def __init__(self, message=None):
        self.message = message
foo = Foo('hello world')
foo.child = foo
try:
    print(' foo.child.child.child.message=', foo.child.child.child.message)
except AttributeError:
    print('AttributeError') 
EOF
$ python3 foo.py
\end{lstlisting}


\filbreak
\begin{problem}
    Write the output of the final command in the following terminal session.
    If the command has no output, then leave the problem blank.
\end{problem}
\begin{lstlisting}
$ cd; rm -rf quiz; mkdir quiz; cd quiz
$ cat > foo.py <<EOF
class Foo:
    def __init__(self, message=None):
        self.message = message
foo = Foo('hello world')
foo.child = Foo('hola mundo')
foo.child.child = foo
try:
    print(' foo.child.child.child.message=', foo.child.child.child.message)
except AttributeError:
    print('AttributeError') 
EOF
$ python3 foo.py
\end{lstlisting}


\filbreak
\begin{problem}
    Write the output of the final command in the following terminal session.
    If the command has no output, then leave the problem blank.
\end{problem}
\begin{lstlisting}
$ cd; rm -rf quiz; mkdir quiz; cd quiz
$ cat > foo.py <<EOF
class Foo:
    def __init__(self, message=None):
        self.message = message
foo = Foo('hello world')
foo.child = Foo('hola mundo')
foo.child.child = foo.child
try:
    print(' foo.child.child.child.message=', foo.child.child.child.message)
except AttributeError:
    print('AttributeError') 
EOF
$ python3 foo.py
\end{lstlisting}

\subsection{With built-in classes}

\filbreak
\begin{problem}
    Write the output of the final command in the following terminal session.
    If the command has no output, then leave the problem blank.
\end{problem}
\begin{lstlisting}
$ cd; rm -rf quiz; mkdir quiz; cd quiz
$ cat > foo.py <<EOF
xs = [1, 2, 3]
xs.append(xs)
xs[0] = xs
try:
    print('xs[-1][0][2]=',xs[-1][0][2])
except TypeError:
    print('TypeError')
EOF
$ python3 foo.py
\end{lstlisting}

\filbreak
\begin{problem}
    Write the output of the final command in the following terminal session.
    If the command has no output, then leave the problem blank.
\end{problem}
\begin{lstlisting}
$ cd; rm -rf quiz; mkdir quiz; cd quiz
$ cat > foo.py <<EOF
xs = [1, 2, 3]
xs.append(xs)
xs[0] = xs
try:
    print('xs[1][0][2]=',xs[1][0][2])
except TypeError:
    print('TypeError')
EOF
$ python3 foo.py
\end{lstlisting}

\filbreak
\begin{problem}
    Write the output of the final command in the following terminal session.
    If the command has no output, then leave the problem blank.
\end{problem}
\begin{lstlisting}
$ cd; rm -rf quiz; mkdir quiz; cd quiz
$ cat > foo.py <<EOF
class Foo(list):
    def __init__(self, xs=[]):
        xs.append(len(xs))
        super().__init__(xs)
xs = Foo()
xs.append(Foo())
xs[0] = 1
xs.append(Foo())
try:
    print('xs[0]=', xs[0])
except AttributeError:
    print('AttributeError') 
EOF
$ python3 foo.py
\end{lstlisting}

%%%%%%%%%%%%%%%%%%%%%%%%%%%%%%%%%%%%%%%%%%%%%%%%%%%%%%%%%%%%%%%%%%%%%%%%%%%%%%%%
\end{document}
