\documentclass[10pt]{article}

\usepackage[margin=1in]{geometry}
\usepackage{amsmath}
\usepackage{amssymb}
\usepackage{amsthm}
\usepackage{mathtools}
\usepackage[shortlabels]{enumitem}
\usepackage[normalem]{ulem}
\usepackage{courier}

\usepackage{hyperref}
\hypersetup{
  colorlinks   = true, %Colours links instead of ugly boxes
  urlcolor     = black, %Colour for external hyperlinks
  linkcolor    = blue, %Colour of internal links
  citecolor    = blue  %Colour of citations
}

\usepackage[T1]{fontenc}
\usepackage{listings}
\lstset{
    language=HTML
    ,basicstyle=\linespread{1}\ttfamily
    ,keywordstyle=
    %,numbers=left
    ,breaklines=true
    }

%%%%%%%%%%%%%%%%%%%%%%%%%%%%%%%%%%%%%%%%%%%%%%%%%%%%%%%%%%%%%%%%%%%%%%%%%%%%%%%%

\theoremstyle{definition}
\newtheorem{problem}{Problem}
\newtheorem{note}{Note}
\newcommand{\E}{\mathbb E}
\newcommand{\R}{\mathbb R}
\DeclareMathOperator{\Var}{Var}
\DeclareMathOperator*{\argmin}{arg\,min}
\DeclareMathOperator*{\argmax}{arg\,max}

\newcommand{\trans}[1]{{#1}^{T}}
\newcommand{\loss}{\ell}
\newcommand{\w}{\mathbf w}
\newcommand{\mle}[1]{\hat{#1}_{\textit{mle}}}
\newcommand{\map}[1]{\hat{#1}_{\textit{map}}}
\newcommand{\normal}{\mathcal{N}}
\newcommand{\x}{\mathbf x}
\newcommand{\y}{\mathbf y}
\newcommand{\ltwo}[1]{\lVert {#1} \rVert}

%%%%%%%%%%%%%%%%%%%%%%%%%%%%%%%%%%%%%%%%%%%%%%%%%%%%%%%%%%%%%%%%%%%%%%%%%%%%%%%%

\begin{document}
\begin{center}
    {
\Large
    Quiz: Counting (Practice Problems)
}

    \vspace{0.1in}
\end{center}

\vspace{0.15in}
\noindent
%\textbf{Note:}
\begin{note}
The format of this quiz will be the same as the previous quiz (4 problems, each worth 1 point).
1 of these problems will be taken from the material from topic 0,
and the other 3 problems will be on functions following the problems below.
\end{note}
\vspace{0.15in}

\section{Nested for loops}

\filbreak
\begin{problem}
    Write the output of the final command in the following terminal session.
    If the command has no output, then leave the problem blank.
\end{problem}
\begin{lstlisting}
$ cd; rm -rf quiz; mkdir quiz; cd quiz
$ N=8
$ cat > foo.py <<EOF
count = 0
for i in range($N):
    count += 1
print('count=', count)
EOF
$ python3 foo.py
\end{lstlisting}
\vspace{0.4in}

\filbreak
\begin{problem}
    Write the output of the final command in the following terminal session.
    If the command has no output, then leave the problem blank.
\end{problem}
\begin{lstlisting}
$ cd; rm -rf quiz; mkdir quiz; cd quiz
$ N=8
$ cat > foo.py <<EOF
count = 0
for i in range($N):
    for j in range($N):
        count += 1
print('count=', count)
EOF
$ python3 foo.py
\end{lstlisting}
\vspace{0.4in}

\filbreak
\begin{problem}
    Write the output of the final command in the following terminal session.
    If the command has no output, then leave the problem blank.
\end{problem}
\begin{lstlisting}
$ cd; rm -rf quiz; mkdir quiz; cd quiz
$ N=8
$ cat > foo.py <<EOF
count = 0
for i in range($N):
    for j in range($N):
        for k in range($N):
            count += 1
print('count=', count)
EOF
$ python3 foo.py
\end{lstlisting}
\vspace{0.1in}

\filbreak
\section{Sequential for loops}

\begin{problem}
    Write the output of the final command in the following terminal session.
    If the command has no output, then leave the problem blank.
\end{problem}
\begin{lstlisting}
$ cd; rm -rf quiz; mkdir quiz; cd quiz
$ N=8
$ cat > foo.py <<EOF
count = 0
for i in range($N):
    count += 1
for i in range($N):
    count += 1
print('count=', count)
EOF
$ python3 foo.py
\end{lstlisting}
\vspace{0.1in}

\filbreak
\begin{problem}
    Write the output of the final command in the following terminal session.
    If the command has no output, then leave the problem blank.
\end{problem}
\begin{lstlisting}
$ cd; rm -rf quiz; mkdir quiz; cd quiz
$ N=8
$ cat > foo.py <<EOF
count = 0
for i in range($N):
    count += 1
for i in range($N):
    count += 1
for i in range($N):
    count += 1
print('count=', count)
EOF
$ python3 foo.py
\end{lstlisting}
\vspace{0.4in}

\filbreak
\section{Combinations of sequential and nested loops}

\begin{problem}
    Write the output of the final command in the following terminal session.
    If the command has no output, then leave the problem blank.
\end{problem}
\begin{lstlisting}
$ cd; rm -rf quiz; mkdir quiz; cd quiz
$ N=8
$ cat > foo.py <<EOF
count = 0
for i in range($N):
    for j in range($N):
        count += 1
for i in range($N):
    count += 1
    count += 1
count += 1
print('count=', count)
EOF
$ python3 foo.py
\end{lstlisting}
\vspace{0.4in}

\filbreak
\begin{problem}
    Write the output of the final command in the following terminal session.
    If the command has no output, then leave the problem blank.
\end{problem}
\begin{lstlisting}
$ cd; rm -rf quiz; mkdir quiz; cd quiz
$ N=8
$ cat > foo.py <<EOF
count = 0
for i in range($N):
    count += 1
    for j in range($N):
        count += 1
for i in range($N):
    count += 1
print('count=', count)
EOF
$ python3 foo.py
\end{lstlisting}
\vspace{0.4in}

\filbreak
\begin{problem}
    Write the output of the final command in the following terminal session.
    If the command has no output, then leave the problem blank.
\end{problem}
\begin{lstlisting}
$ cd; rm -rf quiz; mkdir quiz; cd quiz
$ N=8
$ cat > foo.py <<EOF
count = 0
for i in range($N):
    count += 1
for i in range($N):
    count += 1
    for j in range($N):
        count += 1
    count += 1
    for j in range($N):
        count += 1
        for k in range($N):
            count += 1
for i in range($N):
    count += 1
print('count=', count)
EOF
$ python3 foo.py
\end{lstlisting}
\vspace{0.4in}

\filbreak
\section{``Simple'' Expressions in Multiple Variables}

\begin{problem}
    Write the output of the final command in the following terminal session.
    If the command has no output, then leave the problem blank.
\end{problem}
\begin{lstlisting}
$ cd; rm -rf quiz; mkdir quiz; cd quiz
$ N=8
$ M=4
$ cat > foo.py <<EOF
count = 0
for i in range($N):
    for j in range($M):
        count += 1
print('count=', count)
EOF
$ python3 foo.py
\end{lstlisting}
\vspace{0.4in}

\filbreak
\begin{problem}
    Write the output of the final command in the following terminal session.
    If the command has no output, then leave the problem blank.
\end{problem}
\begin{lstlisting}
$ cd; rm -rf quiz; mkdir quiz; cd quiz
$ N=8
$ M=4
$ cat > foo.py <<EOF
count = 0
for i in range($N):
    count += 1
for j in range($M):
    count += 1
print('count=', count)
EOF
$ python3 foo.py
\end{lstlisting}
\vspace{0.4in}

\filbreak
\begin{problem}
    Write the output of the final command in the following terminal session.
    If the command has no output, then leave the problem blank.
\end{problem}
\begin{lstlisting}
$ cd; rm -rf quiz; mkdir quiz; cd quiz
$ N=8
$ M=4
$ cat > foo.py <<EOF
count = 0
for i in range($N):
    count += 1
    for j in range($M):
        count += 1
    for j in range($N):
        count += 1
for i in range($M):
    count += 1
    for j in range($M):
        count += 1
        count += 1
        count += 1
print('count=', count)
EOF
$ python3 foo.py
\end{lstlisting}
\vspace{0.4in}

\filbreak
\begin{problem}
    Write the output of the final command in the following terminal session.
    If the command has no output, then leave the problem blank.
\end{problem}
\begin{lstlisting}
$ cd; rm -rf quiz; mkdir quiz; cd quiz
$ N=8
$ M=4
$ O=2
$ cat > foo.py <<EOF
count = 0
for i in range($N):
    count += 1
    for j in range($M):
        count += 1
    for j in range($O):
        count += 1
for i in range($M):
    count += 1
    for j in range($O):
        count += 1
        count += 1
        count += 1
print('count=', count)
EOF
$ python3 foo.py
\end{lstlisting}
\vspace{0.4in}

\filbreak
\section{``Complex'' Expressions in Multiple Variables}

\begin{problem}
    Write the output of the final command in the following terminal session.
    If the command has no output, then leave the problem blank.
\end{problem}
\begin{lstlisting}
$ cd; rm -rf quiz; mkdir quiz; cd quiz
$ N=8
$ M=4
$ O=2
$ cat > foo.py <<EOF
count = 0
for i in range($N * $M):
    for j in range($M * $O):
        count += 1
print('count=', count)
EOF
$ python3 foo.py
\end{lstlisting}
\vspace{0.4in}

\filbreak
\begin{problem}
    Write the output of the final command in the following terminal session.
    If the command has no output, then leave the problem blank.
\end{problem}
\begin{lstlisting}
$ cd; rm -rf quiz; mkdir quiz; cd quiz
$ N=8
$ M=4
$ O=2
$ cat > foo.py <<EOF
count = 0
for i in range($N + $M):
    for j in range($M * $O):
        count += 1
print('count=', count)
EOF
$ python3 foo.py
\end{lstlisting}
\vspace{0.4in}

\filbreak
\begin{problem}
    Write the output of the final command in the following terminal session.
    If the command has no output, then leave the problem blank.
\end{problem}
\begin{lstlisting}
$ cd; rm -rf quiz; mkdir quiz; cd quiz
$ N=8
$ M=4
$ O=2
$ cat > foo.py <<EOF
count = 0
for i in range($N + $M):
    count += 1
for i in range($M * $O):
    count += 1
print('count=', count)
EOF
$ python3 foo.py
\end{lstlisting}
\vspace{0.4in}

\filbreak
\begin{problem}
    Write the output of the final command in the following terminal session.
    If the command has no output, then leave the problem blank.
\end{problem}
\begin{lstlisting}
$ cd; rm -rf quiz; mkdir quiz; cd quiz
$ N=8
$ M=4
$ O=2
$ cat > foo.py <<EOF
count = 0
for i in range($N + $M):
    count += 1
    for j in range($N ** $O):
        count += 1
for i in range($M * $O):
    count += 1
print('count=', count)
EOF
$ python3 foo.py
\end{lstlisting}
\vspace{0.4in}

\filbreak
\begin{problem}
    Write the output of the final command in the following terminal session.
    If the command has no output, then leave the problem blank.
\end{problem}
\begin{lstlisting}
$ cd; rm -rf quiz; mkdir quiz; cd quiz
$ N=8
$ M=4
$ O=2
$ cat > foo.py <<EOF
count = 0
for i in range($N + $M):
    count += 1
for i in range($N ** $O):
    count += 1
    for j in range($M + $O):
        count += 1
print('count=', count)
EOF
$ python3 foo.py
\end{lstlisting}
\vspace{0.4in}

\filbreak

\begin{note}
    The only math operations I will use in the real quiz are addition, multiplication, and exponentiation.
\end{note}

\end{document}
